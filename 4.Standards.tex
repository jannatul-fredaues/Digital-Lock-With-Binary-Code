\chapter{Engineering Standards and Mapping}
%This is for reference only. Delete before finalization
The Simple Combination Lock project adheres to basic engineering standards by ensuring electrical safety, circuit reliability, and user accessibility. The design respects key practices in digital logic engineering, including proper logic gate implementation, current-limiting resistors, protection for CMOS components, and intuitive user interface through DIP switches and visual indicators (LEDs). System design and testing were carried out in line with educational objectives, focusing on logic gate application and circuit behavior under standard operating conditions.

\section{Social and Environmental Sustainability}
The Simple Combination Lock promotes technological awareness and security-based automation by offering a simple, logic-driven access control system suitable for low-scale applications such as lockers, drawers, and educational experiments. Socially, it enhances personal security and accessibility, particularly for the elderly or individuals with physical limitations, by eliminating the need for traditional keys and enabling easier control of secured spaces. It also serves as an educational tool, helping students and enthusiasts understand digital logic and foster interest in STEM fields. Environmentally, the system consumes minimal power due to its CMOS components and LED indicators, but its use of electronic parts like ICs and plastic DIP switches raises concerns about long-term sustainability. To address this, responsible e-waste management and component recycling are essential, ensuring the system supports energy-efficient and eco-conscious design principles even at a small scale.\cite{4.1}


\subsection{Impact on Life}
\textbf {Hands-Free Code Entry:} Simple Combination Locks make managing access to small secured spaces like lockers, drawers, or cabinets more convenient. With no need for physical keys, users can quickly unlock systems using DIP switch codes, making daily tasks smoother. This system is commonly usable in home, school, and office environments, offering easy and reliable locking. Quick code entry saves time when managing multiple secured compartments. Since the lock is based on simple logic gates and uses a basic push-button activation, it becomes an effective solution for minimal interaction use. It works instantly upon code match and reduces the effort required to operate mechanical locks.\\
\newline
\textbf{Support for Elderly and People with Disabilities:}
For people with physical limitations or disabilities, this lock provides an easy and comfortable solution. There is no need to twist keys or apply force—just flip DIP switches to set or enter a code. Elderly individuals benefit from reduced physical strain and improved independence in managing personal belongings securely. In settings like labs or dim environments, the lock helps users avoid fumbling with traditional locks, adding safety and convenience. It can be used as part of accessibility-enhancing setups, giving users better control over secured access.\\
\newline
\textbf{Challenges and Limitations:}
The main challenge with this system is its simplicity. A 4-bit code provides only 16 possible combinations, which makes the system vulnerable to guesswork. Anyone could try all combinations within a short time if no physical deterrents are in place. Also, users unfamiliar with binary switches might find it confusing at first. Another issue is that the physical DIP switches can wear out with heavy use. The lock only controls one action (Go or No-Go), making it unsuitable for more complex security systems unless expanded. While effective for learning and basic security, it may not meet the needs of high-risk environments without added layers of protection.\cite{4.1.1}

\subsection{Impact on Society \& Environment}
Simple Combination Lock systems, built using basic digital logic gates, have a significant impact on both social utility and environmental consciousness. These devices improve accessibility and offer an easy way for users to secure their belongings without the need for physical keys or complex mechanisms. This feature is especially beneficial in educational institutions, shared workplaces, and public labs, where quick and easy access control fosters better organization and security. For individuals with physical limitations, it supports independence and control over personal storage areas, helping promote inclusivity and reduced reliance on others. The lock design encourages awareness of digital logic among students and early learners, contributing positively to technical education and interest in STEM fields.\\
From an environmental perspective, the circuit uses low-power CMOS ICs and energy-efficient LEDs, helping to reduce power consumption compared to traditional mechanical or electronic lock systems that rely on motors or high-current devices. However, like most small-scale electronics, the use of plastic DIP switches, semiconductors, and disposable batteries raises concerns about long-term sustainability. Improper disposal of these components could contribute to e-waste and resource depletion. Promoting proper e-waste recycling and designing with replaceable or reusable parts can help address this. Encouraging students and users to reuse breadboards, switches, and ICs for other projects can also reduce overall waste. While the system is simple, its widespread use and educational relevance make it a useful entry point for promoting responsible electronics use and sustainability habits.\cite{4.1.2}
\subsection{Ethical Aspects}
The ethical considerations of the Simple Combination Lock revolve around accessibility, educational value, responsible production, and long-term usability. By offering a low-cost, easy-to-use locking system, it ensures that individuals from various backgrounds—especially students or low-income users—can access basic security solutions. This supports technological inclusivity and promotes a sense of independence, especially for vulnerable groups like the elderly or disabled. Ethically, this aligns with the goal of building technology that serves broad communities rather than exclusive segments of society.\\
Privacy is not a major concern in this design since it doesn’t involve data collection or communication systems. However, users must understand the limitations of the system’s simplicity—its 4-bit code offers only 16 possible combinations, which can be cracked easily through trial and error. There is an ethical responsibility to inform users about these limitations and avoid overpromising the level of security provided. On the production side, environmental responsibility also plays a role. The project should use responsibly sourced materials and encourage recycling of components like ICs and batteries to minimize harmful e-waste. Additionally, educators and developers should guide users in using the device ethically and not misapplying it for deceptive or unauthorized security bypass attempts. Responsible use, combined with clear educational guidance, ensures that this simple system promotes ethical awareness in both technical and social contexts.\cite{4.1.3}
\subsection{Sustainability Plan}
To ensure sustainability throughout the lifecycle of the Simple Combination Lock, a plan must integrate environmental, social, and economic aspects. During the design stage, the focus should be on using commonly available, reusable materials such as CMOS ICs (4001, 4070), resistors, and DIP switches, which can be disassembled and reused in other circuits. The use of energy-efficient LEDs and low-power components also contributes to eco-friendly operation. Breadboard-based implementations are ideal for education and experimentation, reducing the need for permanent soldering and extending the life of parts.\\
From a manufacturing perspective, if scaled, small PCBs can be produced using low-impact processes and recyclable substrates. Avoiding excessive plastic casing and opting for modular enclosures made from recycled or biodegradable material helps reduce environmental impact. In distribution, parts can be bundled in compact packaging using paper-based or recyclable containers. Educational kits should include instructions on reusability and e-waste disposal to raise awareness.\\
To promote sustainability during use, users should be encouraged to operate the lock using rechargeable batteries and turn off the system when not in use to minimize standby power consumption. At end-of-life, components should be separated and recycled properly, and institutions can establish collection points for student-used components. Economically, the system remains highly affordable—ideal for educational use—and scales well for batch production in low-cost kits. By ensuring responsible material use, easy repairability, and promoting awareness of sustainability through education, the Simple Combination Lock project can deliver long-term value with minimal environmental burden.\cite{4.1.4}

%\section

\section{Project Management and Team Work}
This project was completed through a collaborative effort involving three team members, each taking on a distinct responsibility to ensure the successful development of the Simple Combination Lock. One team member was responsible for managing the procurement of electronic components, handling the project budget, and ensuring the timely availability of all necessary materials, thereby keeping the project within cost constraints and maintaining component quality. Another team member focused on designing and assembling the circuit on a breadboard, testing the logic, and troubleshooting any issues to ensure the functionality and reliability of the lock mechanism. The third member took charge of preparing comprehensive documentation, including project reports, circuit diagrams, cost analysis, and presentation materials, which helped maintain clear records and facilitated easy understanding and replication of the project. This clear division of responsibilities allowed for efficient project execution and effective communication among team members, providing valuable experience in collaborative engineering, project management, and practical problem-solving.\cite{4.2}

\section{Complex Engineering Problem}
\textbf{Problem Statement:}
Manual access control methods like traditional key locks can be inconvenient, especially for users with physical impairments or in environments where hands-free or simplified access is preferred. Conventional locks require precision and mechanical interaction, which can be frustrating in tight, dark, or high-turnover areas like labs, workshops, or shared storage facilities.\\
\textbf{The suggested Solution:}
The Simple Combination Lock is an accessible and logical alternative that uses digital electronics to regulate access. It allows a secure code to be entered via DIP switches, and uses XOR logic to compare the entered code with a preset combination. When the correct code is entered, a signal activates the “Go” LED, while incorrect codes trigger a “No Go” signal, providing immediate feedback.\\
\textbf{Key Features of the Solution:}
\begin{itemize}
\item Compares code inputs using XOR gates to ensure access only when all bits match.
\item Uses NOR logic to control output LEDs based on real-time code verification.
\item Requires no microcontroller or programming, making it ideal for beginners and educational use.
\item Fully passive and low-power, suitable for battery operation.
\item Easy to assemble and cost-efficient, promoting wide usability.
\item Adaptable to future upgrades like sound-based input, digital keypads, or smart integration.\cite{4.3}
\end{itemize}

\subsection{Mapping of Program Outcome} 
\textbf{Problem Statement:}
The manually operation of electrical appliances can be awkward, particularly for People with physical impairments or in circumstances where hands-free command is desirable. Traditional switches need direct make contact with .which can be cumbersome in specific scenarios.\\
\textbf{The suggested Solution:}
A clap switch is an creative device that permits electrical gadgets to be regulated using sound, specifically claps. The system recognizes the acoustic signal of a clap and Activators a relay to switch the linked device on or off. This provides a hands-free, practical, and accessible substitute to traditional switches.\\
\textbf{Key Features of the Solution:}
\begin{itemize}
\item Identifies clap sounds using a microphone and sound identification circuit.
\item Procedures the signal using an amplifier and transistor-based circuit to turn on a relay.
\item Easy, low-cost implementation with minimal upkeep.
\item Perfect for smart home applications and Availability solutions.\cite{4.3.1}
\end{itemize}

\begin{center}
    \begin{table}[ht]
    
        \begin{tabular}{|p{0.2\textwidth}|p{0.7\textwidth}|}
            \hline
            \textbf{PO's} & \textbf{Justification} \\
            \hline
            PO1 & Exhibits the application of Digital Logic Design principles in circuit creation. \\
            \hline
            PO2 & Determines user requirements and of Emulates a practical solution with digitally solve for command. \\
            \hline
            PO3 & Designs and tools a functional digital lock with binary code real-world Relevance. \\
            \hline
        \end{tabular}
        \centering
        \caption{Justification of Program Outcomes.}
        \label{tab:po_justification}
    \end{table}
\end{center}

\begin{center}
    \begin{table}[ht]
    \subsection{Complex Problem Solving} 
        \begin{tabular}{|p{0.12\textwidth}|p{0.12\textwidth}|p{0.12\textwidth}|p{0.12\textwidth}|p{0.12\textwidth}|p{0.12\textwidth}|p{0.12\textwidth}|}
        \hline
        EP1& EP2& EP3& EP4& EP5& EP6& EP7\\
        Dept of Knowledge & Range of Conflicting Requirements & Depth of Analysis & Familiarity of Issues & Extent of Applicable Codes & Extent of Stakeholder Involvement & Inter-dependence\\
        \hline 
         The project requires knowledge of electronics, microphone technology and signal processing to develop a reliable digital lock system. Key challenges include balancing sensitivity and reliability, with a focus on minimizing false positives. Engineers can manage these issues with microcontroller and signal processing expertise.
&The project requires that every problem was solving process is digitally try to solve. Some key skills included programming, signal handling, algorithm design and microcontroller use.
 &The project requires moderate hardware integration, focusing on reliable component functionality.  Some key skills included circuit design, switch button press and so more. Balancing cost and performance is essential, addressing challenges like power supply.
&Testing involves common challenges like debugging hardware failures and adjusting errors. The system must balance performance with user expertise, being both sensitive and robust in varied conditions. 
&The project requires Digital Lock with Binary Code follows standards like IEEE 91-1984 for digital circuit design and symbol representation. 
&This project shows that users, developers and supervisors are involved in defining requirements, testing functionality and ensuring usability of the digital lock. 
&The project demonstrates interdependence between hardware components, where the binary input system relies on both circuit logic. Team collaboration across design, coding and testing phases ensures smooth integration and functionality.\\
        &&&&&&\\
        \hline 
        \end{tabular}
         \centering
        \caption{Mapping with complex problem-solving.}
        \label{tab:p_solve}
    \end{table}
     
\end{center}

\begin{center}
    \begin{table}[ht]
    \subsection{Engineering Activities}
    
        \begin{tabular}{|p{0.18\textwidth}|p{0.18\textwidth}|p{0.18\textwidth}|p{0.18\textwidth}|p{0.18\textwidth}|}
        \hline
        EA1& EA2& EA3& EA4& EA5\\
        Range of resources & Level of Interaction & Innovation & Consequences for society and environment & Familiarity\\
        \hline 
         Simple resources, materials, a microcontroller and software tools are utilized. In the middle engagement, cooperation between hardware and software developers is essential. Typical components like keypads and microcontrollers are used, but the creative part lies in the algorithm design for recognizing and validating the correct binary code.
         &Low engagement since component choice can be a solo assignment, but may need cooperation with suppliers or hardware suppliers. Accessibility of components in the local marketplace.

         &Creative algorithms or signal processing methods may be needs to detect the clap from background sound. Applications software tools for modeling and examination, along with physical components: High engagement between hardware and software features, as the microcontroller programming must be in agreement with the circuit design.
         &Possibility for product endurance and energy efficiency, which can have positive social effects. Environmental issues are minimal if proper recycling practices are followed for electronic components. Requires tools for debugging and an appropriate testing environment. High potential for innovation, as the binary code validation algorithm can lead to new techniques for enhanced system reliability.
         &The project builds on basic knowledge of digital electronics, binary number systems and logic gates, making it familiar to students and engineers in related fields. Its simple interface and concept make it easy to understand, implement and customize.\\
        &&&&\\
        \hline 
        \end{tabular}
        \centering
                \caption{Mapping with complex engineering activities.}
        \label{tab:e_act}
    \end{table}
\end{center}
